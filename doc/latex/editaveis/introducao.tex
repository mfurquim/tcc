\chapter[Introdução]{Introdução}
\label{introduc}

\section{Contextualização e Justificativa}

Por alguns anos a teoria dos jogos vem estudando o comportamento de
indivíduos sob uma situação de conflito, como em jogos, balança de
poder, leilões, e até mesmo evolução
genética \cite{sartini_IIbienaldasbm}. Esta área possui duas frentes de
estudo: (a) \emph{teoria econômica dos jogos}, o qual possui motivações
predominante econômicas, e (b) \emph{teoria combinatória dos jogos}, que
faz uso dos aspectos combinatórios de jogos de mesa e não permite
elementos imprevisíveis.

Este trabalho faz uso de ambas abordagens, sendo que, a partir da
primeira abordagem, é estabelecido um método para se maximizar o ganho
(\emph{payoff}) e, a partir da segunda, um método para identificar uma
jogada que garantirá a vitória independente do resto do jogo.

O restante deste trabalho está organizado da seguinte maneira: Na seção
\ref{cha:fundamentacao-teorica} é narrado uma breve
história da teoria dos jogos e seus conceitos fundamentais, além de
conter explicação para os temas de análise de complexidade, análise
combinatória e programação dinâmica, e explicação das regras do jogo
\emph{Big Points}. A seção seguinte (\ref{cha:metodologia}) lista os
equipamentos, \textit{softwares} e metodologia utilizados para o desenvolvimento
do trabalho e, também, a maneira que a foi analisado o jogo. Os resultados,
até o momento, são descritos na seção \ref{cha:resultados-parciais}, o
cronograma de trabalho na seção \ref{cha:cronograma}, e as considerações
finais na seção \ref{cha:consideracoes_finais}.
