\chapter[Metodologia]{Metodologia}
\label{cha:metodologia}
	Este capítulo descreve os passos para a realização deste trabalho, explicando os equipamentos e softwares utilizados.

\section{Levantamento Bibliográfico}
\label{sec:levantamento-bibliografico}
Após a definição do tema, foi realizado uma pesquisa a respeito dos conceitos básicos da Teoria dos Jogos, a existência de trabalhos semelhantes e materiais suficientes para a realização deste trabalho.


\section{Lista de Equipamentos e Softwares}
\label{sec:lista-de-equipamentos-e-softwares}

Para a realização deste trabalho foi utilizado um computador da linha \emph{Inspiron 14Rx} fabricado pela \emph{Dell}, no qual possui processador \emph{Intel Core i7} de 2,2 GHz, GPU \emph{NVIDIA GeForce GT 630M} de 1GB e 8GB de memória RAM. Quanto aos softwares utilizados, foram apenas os pacotes básicos de desenvolvimento em C/C++, incluindo \emph{GCC} e \emph{GNU Makefile}. Para serviço de versionamento foi utilizado o \emph{GitHub} e para controle de tarefas e \emph{issues} foi utilizado o \emph{waffle.io}.



\begin{itemize}
\item
  O jogo eletrônico está sendo implementado
\item
  Foi feito uma análise para descobrir a possibilidade de computar a
  melhor jogada possível para um ou vários jogos. Nesta análise,
  levou-se em consideração:

  \begin{itemize}
  \tightlist
  \item
    A quantidade de memória necessária
  \item
    O número de estados existentes
  \item
    A complexidade assintótica do algoritmo (pois o número de entrada
    pro algorítimo é muito grande)
  \item
    O tempo de processamento para um determinado número de estados
  \end{itemize}
\end{itemize}
