\chapter[Metodologia]{Metodologia}
\label{cha:metodologia}
	Este capítulo descreve os passos para a realização deste trabalho, explicando os equipamentos e softwares utilizados.

\section{Levantamento Bibliográfico}
\label{sec:levantamento-bibliografico}
Após a definição do tema, foi realizado uma pesquisa a respeito dos conceitos básicos da Teoria dos Jogos, a existência de trabalhos semelhantes e materiais suficientes para a realização deste trabalho.

\section{Lista de Equipamentos e Softwares}
\label{sec:lista-de-equipamentos-e-softwares}

Para a realização deste trabalho foi utilizado um computador da linha \emph{Inspiron 14Rx} fabricado pela \emph{Dell}, no qual possui processador \emph{Intel Core i7} de 2,2 GHz, GPU \emph{NVIDIA GeForce GT 630M} de 1GB e 8GB de memória RAM. Quanto aos softwares utilizados, foram apenas os pacotes básicos de desenvolvimento em C/C++, incluindo \emph{GCC} e \emph{GNU Makefile}. Para serviço de versionamento foi utilizado o \emph{GitHub} e para controle de tarefas e \emph{issues} foi utilizado o \emph{waffle.io}.

\section{Análise combinatória}

A possibilidade de encontrar a solução para este jogo depende da quantidade de estados existentes e da quantidade de espaço cada estado ocupará na memória. Se for possível calcular para uma partida, foi feito uma análise combinatória para descobrir a possibilidade de encontrar uma solução para várias partidas distintas.

\subsection{Número de partidas distintas}
As características importantes que distinguem uma partida da outra no jogo \emph{Big Points} são: (a) a quantidade de jogadores e; (b) a ordem dos discos no tabuleiro. O jogo pode ser jogado de dois a cinco jogadores ($J$) e o tabuleiro possui 55 discos no total ($D_T$), 5 das cores \emph{branco} ($D_W$) e \emph{preto} ($D_K$), e 9 das cores \emph{Vermelho} ($D_R$), \emph{Verde} ($D_G$), \emph{Azul} ($D_B$), \emph{Amarelo} ($D_Y$) e \emph{Violeta} ($D_V$). Com isso, tem-se que a quantidade de jogos distintos é a combinação dos discos e dos jogadores, como demonstrado na equação \ref{eq_partidas}, resultando em um número maior do que $5\times 10^{41}$.

Considerando $\#D_{L1} = \#D_T - \#D_W$ como o restante dos discos após a primeira combinação, $\#D_{L2} = \#D_{L1} - \#D_K$ o restante da segunda combinação, e assim sucessivamente, temos a equação \ref{eq_partidas}.

 \begin{equation}
	 \label{eq_partidas}
	 \tag{e.q. Números de Partidas Distintas}
 \begin{split}
 Partidas\ &=\  (\#J-1) \cdot \binom{\#D_T}{\#D_W} \cdot \binom{\#D_{L1}}{\#D_K} \cdot \binom{\#D_{L2}}{\#D_R} \cdot \binom{\#D_{L3}}{\#D_G} \cdot \binom{\#D_{L4}}{\#D_B} \cdot \binom{\#D_{L5}}{\#D_Y} \cdot \binom{\#D_{L6}}{\#D_V}\\
 Partidas\ &=\  4\cdot \binom{55}{5} \cdot \binom{50}{5} \cdot \binom{45}{9} \cdot \binom{36}{9} \cdot \binom{27}{9} \cdot \binom{18}{9} \cdot \binom{9}{9}\\
 Partidas\ &=\ 560'483'776'167'774'018'942'304'261'616'685'408'000'000\\
 Partidas\ &\approx 5\times 10^{41}
 \end{split}
 \end{equation}

Considerando a possibilidade de solucionar uma partida do jogo por segundo, este cálculo levaria mais do que $10^{34}$ anos, como demonstrado na equação \ref{eq_anos_partidas}.

 \begin{equation} \label{eq_anos_partidas} \tag{e.q. Tempo de Computação das Partidas}
 \begin{split}
 Anos\ &=\ \dfrac{N_{partidas\ distintas}}{\nicefrac{partida}{segundo}\cdot\nicefrac{segundos}{minuto}\cdot\nicefrac{minutos}{hora}\cdot\nicefrac{horas}{dia}\cdot\nicefrac{dias}{ano}}\\
 Anos\ &=\ \dfrac{560'483'776'167'774'018'942'304'261'616'685'408'000'000}{\nicefrac{1}{1}\cdot\nicefrac{60}{1}\cdot\nicefrac{60}{1}\cdot\nicefrac{24}{1}\cdot\nicefrac{365}{1}}\\
 Anos\ &=\ 96'526'964'154'064'571'465'728\\
 Anos\ &\approx 9\times 10^{34}
 \end{split}
 \end{equation}

\subsection{Número de estados}

Dado uma partida inicial $\gammaup \in \Gamma$ de \emph{Big Points}, sendo $\Gamma$ o conjunto contendo todas as $5\times 10^{41}$ partidas distintas, foi calculado o número de estados de uma partida. Para calcular o número de estados, é preciso determinar as características que definem um estado do jogo. De acordo com a seção \ref{subsec:regras-do-jogo}, tem-se que essas características são: (a) o estado do tabuleiro; (b) o estado dos peões; (c) o estado da escada; (d) o estado dos discos na mão dos jogadores e; (e) o jogador atual. Como demonstrado na equação \ref{eq_estados}, a quantidade de estados existentes para uma partida, considerando um caso geral, é maior do que $3\times 10^{30}$, levando mais de $9\times 10^{22}$ anos para calcular todos os estados a uma velocidade de $1 \nicefrac{E_{stado}}{S_{egundo}}$.

\begin{equation} \label{eq_estados} \tag{e.q. Caso Geral}
\begin{split}
Estados\ &=\ 2^{55}\cdot 61^{5}\cdot (10^{5} + 6^{2})\\
Estados\ &=\ 3'044'074'341'562'580'507'100'339'765'248\\
Estados\ &\approx 3\times 10^{31}
\end{split}
\end{equation}

\begin{equation} \label{eq_anos_estados} \tag{e.q. Tempo de Computação dos Estados}
\begin{split}
Anos\ &=\ \dfrac{N_{estados\ distintos}}{\nicefrac{estado}{segundo}\cdot\nicefrac{segundos}{minuto}\cdot\nicefrac{minutos}{hora}\cdot\nicefrac{horas}{dia}\cdot\nicefrac{dias}{ano}}\\
Anos\ &=\ \dfrac{3'044'074'341'562'580'507'100'339'765'248}{\nicefrac{1}{1}\cdot\nicefrac{60}{1}\cdot\nicefrac{60}{1}\cdot\nicefrac{24}{1}\cdot\nicefrac{365}{1}}\\
Anos\ &=\ 96'526'964'154'064'571'465'728
\end{split}
\end{equation}

\subsection{Espaço de armazenamento}

A quantidade de memória necessária para armazenar um \emph{estado} do jogo depende das características que descrevem um \emph{estado}. Como dito anteriormente, o jogo pode ter até cinco jogadores, possui um tabuleiro com 55 discos, uma escada com cinco degraus, e cinco peões no qual a posição varia entre 0 e 60. Considerando que o estado inicial do tabuleiro é aleatório mas conhecido desde o começo da partida e que os discos não mudam de posição, é possível representar o tabuleiro com uma máscara binária indicando se o disco está ou não disponível, $\nicefrac{55\ D_{iscos}}{8\ B_{its}}$. A posição de cada peão pode ser representada por um \emph{char}, pois varia apenas entre 0 e 60, sendo que 0 indica que o peão não está no tabuleiro, 1 a 55 indica a posição que o peão está no tabuleiro e 56 a 60 indica a posição na escada. Por fim, para representar a mão de cada jogador, foi utilizado sete \emph{chars}, cada um indicando a quantidade de discos de uma determinada cor, e como são cinco jogadores, tem-se $5\cdot 7$. Todos esses \emph{bytes} são somados como demonstrado na equanção \ref{eq_bytes}, totalizando $47$ bytes na memória por estado da partida.

\begin{equation} \label{eq_bytes} \tag{e.q. Bytes na memória}
\begin{split}
Bytes\ &=\ \dfrac{55}{8} + 5 + 5\cdot 7\\
Bytes\ &=\ 47\ bytes
\end{split}
\end{equation}

\subsubsection{Podas}

A quantidade de estados distintos para cada partida foi calculada da seguinte maneira: cada um dos cinco possíveis jogadores pode ter entre zero e cinco discos das cores branco e preta, assim como pode ter entre zero e dez discos das cores restantes; cada peão pode estar em uma posição entre zero e dez (considerando apenas os de sua cor); e, por fim, cada espaço de disco no tabuleiro pode ou não estar ocupado. Partindo destas informações, temos a equação \ref{eq_poda1}.

A posição dos peões pode ser determinada considerando-se apenas as casas do
tabuleiro de sua respectiva cor. Com essa poda, o número de estados distintos
de um jogo é reduzido, mas ainda se encontra na ordem de $10^{21}$.

\begin{equation} \label{eq_poda1} \tag{e.q. Poda por posição}
\begin{split}
Estados\ &=\ 2^{55}\cdot 11\cdot 5\cdot (5\times 10 + 2\cdot 6)\\
Estados\ &=\ 2^{55}\cdot 11\cdot 5\cdot 72\\
Estados\ &=\ 142'674'036'195'097'313'280
\end{split}
\end{equation}

\begin{equation} \label{eq_anos_estados} \tag{e.q. Tempo de Computação dos Estados}
\begin{split}
Anos\ &=\ \dfrac{N_{estados\ distintos}}{\nicefrac{estado}{segundo}\cdot\nicefrac{segundos}{minuto}\cdot\nicefrac{minutos}{hora}\cdot\nicefrac{horas}{dia}\cdot\nicefrac{dias}{ano}}\\
Anos\ &=\ \dfrac{142'674'036'195'097'313'280}{\nicefrac{1}{1}\cdot\nicefrac{60}{1}\cdot\nicefrac{60}{1}\cdot\nicefrac{24}{1}\cdot\nicefrac{365}{1}}\\
Anos\ &=\ 4'524'164'009'230
\end{split}
\end{equation}


\begin{itemize}
\item
  O jogo eletrônico está sendo implementado
\item
  Foi feito uma análise para descobrir a possibilidade de computar a
  melhor jogada possível para um ou vários jogos. Nesta análise,
  levou-se em consideração:

  \begin{itemize}
  \tightlist
  \item
    A quantidade de memória necessária
  \item
    O número de estados existentes
  \item
    A complexidade assintótica do algoritmo (pois o número de entrada
    pro algorítimo é muito grande)
  \item
    O tempo de processamento para um determinado número de estados
  \end{itemize}
\end{itemize}
