\chapter[Metodologia]{Metodologia}
\label{cha:metodologia}
	Este capítulo descreve os passos para a realização deste trabalho, explicando os equipamentos e softwares utilizados.

\section{Levantamento Bibliográfico}
\label{sec:levantamento-bibliografico}
Após a definição do tema, foi realizado uma pesquisa a respeito dos conceitos básicos da Teoria dos Jogos, a existência de trabalhos semelhantes e materiais suficientes para a realização deste trabalho.

\section{Lista de Equipamentos e Softwares}
\label{sec:lista-de-equipamentos-e-softwares}

Para a realização deste trabalho foi utilizado um computador da linha \emph{Inspiron 14Rx} fabricado pela \emph{Dell}, no qual possui processador \emph{Intel Core i7} de 2,2 GHz, GPU \emph{NVIDIA GeForce GT 630M} de 1GB e 8GB de memória RAM. Quanto aos softwares utilizados, foram apenas os pacotes básicos de desenvolvimento em C/C++, incluindo \emph{GCC} e \emph{GNU Makefile}. Para serviço de versionamento foi utilizado o \emph{GitHub} e para controle de tarefas e \emph{issues} foi utilizado o \emph{waffle.io}.

\section{Análise combinatória}

A possibilidade de encontrar a solução para este jogo depende da quantidade de estados existentes e da quantidade de espaço cada estado ocupará na memória. Se for possível calcular para uma partida, foi feito uma análise combinatória para descobrir a possibilidade de encontrar uma solução para várias partidas distintas.

\subsection{Espaço de armazenamento}

A quantidade de memória necessária para armazenar um \emph{estado} do jogo depende das características que descrevem um \emph{estado}. Como dito anteriormente, o jogo pode ter até cinco jogadores, possui um tabuleiro com 55 discos, uma escada com cinco degraus, e cinco peões no qual a posição varia entre 0 e 60. Considerando que o estado inicial do tabuleiro é aleatório mas conhecido desde o começo da partida e que os discos não mudam de posição, é possível representar o tabuleiro com uma máscara binária indicando se o disco está ou não disponível, $\nicefrac{55\ D_{iscos}}{8\ B_{its}}$. A posição de cada peão pode ser representada por um \emph{char}, pois varia apenas entre 0 e 60, sendo que 0 indica que o peão não está no tabuleiro, 1 a 55 indica a posição que o peão está no tabuleiro e 56 a 60 indica a posição na escada. Por fim, para representar a mão de cada jogador, foi utilizado sete \emph{chars}, cada um indicando a quantidade de discos de uma determinada cor, e como são cinco jogadores, tem-se $5\cdot 7$. Todos esses \emph{bytes} são somados como demonstrado na equanção \ref{eq:bytes}, totalizando $47$ bytes na memória por estado da partida.

\begin{equation} \label{eq:bytes} \tag{e.q. Bytes na memória}
\begin{split}
Bytes\ &=\ \dfrac{55}{8} + 5 + 5\cdot 7\\
Bytes\ &=\ 47\ bytes
\end{split}
\end{equation}

\subsection{Quantidade de partidas distintas}
As características importantes que distinguem uma partida da outra no jogo \emph{Big Points} são: (a) a quantidade de jogadores e; (b) a ordem dos discos no tabuleiro. O jogo pode ser jogado de dois a cinco jogadores ($J$) e o tabuleiro possui 55 discos no total ($D_T$), 5 das cores \emph{branco} ($D_W$) e \emph{preto} ($D_K$), e 9 das cores \emph{Vermelho} ($D_R$), \emph{Verde} ($D_G$), \emph{Azul} ($D_B$), \emph{Amarelo} ($D_Y$) e \emph{Violeta} ($D_V$). Com isso, tem-se que a quantidade de jogos distintos é a combinação dos discos e dos jogadores. Considerando $\#D_{L1} = \#D_T - \#D_W$ como o restante dos discos após a primeira combinação, $\#D_{L2} = \#D_{L1} - \#D_K$ o restante da segunda combinação, e assim sucessivamente, temos a equação \ref{eq:partidas}.

 \begin{equation}
	 \label{eq:partidas}
	 \tag{e.q. Quantidades de Partidas Distintas}
 \begin{split}
 Partidas\ &=\  (\#J-1) \cdot \binom{\#D_T}{\#D_W} \cdot \binom{\#D_{L1}}{\#D_K} \cdot \binom{\#D_{L2}}{\#D_R} \cdot \binom{\#D_{L3}}{\#D_G} \cdot \binom{\#D_{L4}}{\#D_B} \cdot \binom{\#D_{L5}}{\#D_Y} \cdot \binom{\#D_{L6}}{\#D_V}\\
 Partidas\ &=\  4\cdot \binom{55}{5} \cdot \binom{50}{5} \cdot \binom{45}{9} \cdot \binom{36}{9} \cdot \binom{27}{9} \cdot \binom{18}{9} \cdot \binom{9}{9}\\
 Partidas\ &=\ 560'483'776'167'774'018'942'304'261'616'685'408'000'000\\
 Partidas\ &\approx 5\times 10^{41}
 \end{split}
 \end{equation}

Como demonstrado na equação \ref{eq:partidas}, a quantidade de partidas distintas é superior a $5\times 10^{41}$. Considerando a possibilidade de solucionar uma partida do jogo por segundo, este resolução levaria mais do que $10^{34}$ anos, como demonstrado na equação \ref{eq:anos_partidas}.

 \begin{equation} \label{eq:anos_partidas} \tag{e.q. Tempo de Computação das Partidas}
 \begin{split}
 Anos\ &=\ \dfrac{N_{partidas\ distintas}}{\nicefrac{partida}{segundo}\cdot\nicefrac{segundos}{minuto}\cdot\nicefrac{minutos}{hora}\cdot\nicefrac{horas}{dia}\cdot\nicefrac{dias}{ano}}\\
 Anos\ &=\ \dfrac{560'483'776'167'774'018'942'304'261'616'685'408'000'000}{\nicefrac{1}{1}\cdot\nicefrac{60}{1}\cdot\nicefrac{60}{1}\cdot\nicefrac{24}{1}\cdot\nicefrac{365}{1}}\\
 Anos\ &=\ 96'526'964'154'064'571'465'728\\
 Anos\ &\approx 9\times 10^{34}
 \end{split}
 \end{equation}

\subsection{Quantidade de estados}
\label{subsec:quantidade-de-estados}

Dado uma partida inicial $\gammaup \in \Gamma$ de \emph{Big Points}, sendo $\Gamma$ o conjunto contendo todas as $5\times 10^{41}$ partidas distintas, foi calculado a quantidade de estados de $\gammaup$. Para realizar este cálculo do número de estados, foi preciso determinar as características que definem um estado do jogo. De acordo com a seção \ref{subsec:regras-do-jogo}, tem-se que essas características são: (a) o estado do tabuleiro; (b) o estado dos peões; (c) o estado da escada; (d) o estado dos discos na mão dos jogadores e; (e) o jogador atual.

\subsubsection{Quantidade de Estados do Tabuleiro}

Já conhecendo a ordem de todos os discos no tabuleiro, o estado do tabuleiro foi descrito por uma máscara binária que representa a disponibilidade destes disco, para um peão mover-se para cima ou para ser pego por um jogador. O jogo possui $55$ discos que podem ou não estar disponíveis, portando existem $55^2$ estados diferentes para o tabuleiro.

\subsubsection{Quantidade de Estados dos Peões}

A próxima característica que define o estado de uma partida é o estado dos peões ou, mais precisamente, a posição deles. Um peão pode não estar no tabuleiro ainda (1), pode estar em cima de qualquer um dos $55$ discos (1--56) e pode estar em cima de algum dos cinco degraus da escada (57--61). Como o jogo possue $5$ peões, a quantidade de estados dos peões são Estado$_{[\text{peões}]}=61^5$.

\subsubsection{Quantidade de Estados da Escada}

O terceiro estado importante é o estado da escada. Inicialmente a escada pode estar vazia, e então pode ter um peão de qualquer cor no topo. Para cada peão no topo da escada, é possível ter qualquer outra cor, que não seja aquela já escolhida, na posição anterior à do topo. Seguindo essa lógica, tem-se que a quantidade de estados da escada é um somatório de arranjos como demonstrado na equação \ref{eq:estado_da_escada}.

\begin{equation} \label{eq:estado_da_escada} \tag{e.q. Estado da escada}
\begin{split}
	Estado_{[escada]}\ &=\ \displaystyle\sum_{i=0}^{5} A_{5,i}\\
	Estado_{[escada]}\ &=\ \dfrac{5!}{(5-0)!} + \dfrac{5!}{(5-1)!} + \dfrac{5!}{(5-2)!} + \dfrac{5!}{(5-3)!} + \dfrac{5!}{(5-4)!} + \dfrac{5!}{(5-5)!}\\
%	Estado_{[escada]}\ &=\ 1 + 5 + 5\cdot4 + 5\cdot4\cdot3 + 5\cdot4\cdot3\cdot2 + 5\cdot4\cdot3\cdot2\cdot1\\
	Estado_{[escada]}\ &=\ 326
\end{split}
\end{equation}

\subsubsection{Quantidade de Estados dos Discos}

A outra característica é o estado dos discos na mão dos jogadores. Entendendo que a quantidade de discos é sempre a mesma ao longo do jogo e que a única coisa que altera é o local do disco (se está no tabuleiro ou na mão de algum jogador), então a seguinte equação sempre deve ser respeitada: $F+J_1+J_2+J_3+J_4+J_5 = 10$. Nessa simples representação, o $F$ corresponde à quantidade de discos que estão fora da mão dos jogadores, $J_1$ à $J_5$ à quantidade de discos na mão dos jogadores de 1 à 5.

Uma outra abordagem para explicar tal estado é imaginar que cada disco corresponde ao caracter \emph{o} e o símbolo \emph{+} separa a mão dos jogadores e tabuleiro. Logo, é fácil de ver que no estado inicial, todos os discos estão no tabuleiro, sendo representado pelo estado $oooooooooo+++++=10$. Caso o primeiro jogador consiga um disco daquela cor, o estado é alterado para $ooooooooo+o++++=10$. Se, posteriormente, o jogador 4 colete 3 discos e o jogador 5 mais 5, o estado seria $o+o+++ooo+ooooo=10$. O cálculo para descobrir a quantidade de estados para cada disco comum é uma combinação de 15 elementos tomados de 5 a 5, como demonstrado na equação \ref{eq:estado_dos_discos_comuns}, e para cada disco especial é a combinação de 10 elementos tomados de 5 a 5, como demonstrado na equação \ref{eq:estado_dos_discos_especiais}.

\begin{equation} \label{eq:estado_dos_discos_comuns} \tag{e.q. Estado dos discos comuns}
\begin{split}
	Estado_{[discos_c]}\ &=\ \binom{15}{5}\\
	Estado_{[discos_c]}\ &=\ \dfrac{15!}{5!(15-5)!}\\
%	Estado_{[discos]}\ &=\ 1 + 5 + 5\cdot4 + 5\cdot4\cdot3 + 5\cdot4\cdot3\cdot2 + 5\cdot4\cdot3\cdot2\cdot1\\
	Estado_{[discos_c]}\ &=\ 3003
\end{split}
\end{equation}

\begin{equation} \label{eq:estado_dos_discos_especiais} \tag{e.q. Estado dos discos especiais}
\begin{split}
	Estado_{[discos_e]}\ &=\ \binom{10}{5}\\
	Estado_{[discos_e]}\ &=\ \dfrac{10!}{5!(10-5)!}\\
%	Estado_{[discos]}\ &=\ 1 + 5 + 5\cdot4 + 5\cdot4\cdot3 + 5\cdot4\cdot3\cdot2 + 5\cdot4\cdot3\cdot2\cdot1\\
	Estado_{[discos_e]}\ &=\ 252
\end{split}
\end{equation}

Ao final, tem-se que a quantidade de estados distintos dos discos é igual ao produto dos estados de cada disco individual, resultando na equação \ref{eq:estado_dos_discos}

\begin{equation} \label{eq:estado_dos_discos} \tag{e.q. Estado dos discos}
\begin{split}
	Estado_{[discos]}\ &=\ \binom{15}{5}\cdot\binom{15}{5}\cdot\binom{15}{5}\cdot\binom{15}{5}\cdot\binom{15}{5}\cdot\binom{10}{5}\cdot\binom{10}{5}\\
	Estado_{[discos]}\ &=\ 3003^5\cdot252^2\\
%	Estado_{[discos]}\ &=\ 1 + 5 + 5\cdot4 + 5\cdot4\cdot3 + 5\cdot4\cdot3\cdot2 + 5\cdot4\cdot3\cdot2\cdot1\\
	Estado_{[discos]}\ &=\ 15'508'783'829'111'892'791'472\\
	Estado_{[discos]}\ &\approx 1\times 10^{22}
\end{split}
\end{equation}

\subsubsection{Quantidade de Estados do Jogador}

A quantidade de estados distintos para o jogador atual é simplesmente a quantidade de jogadores, determinando assim a vez do jogador: Estado$_{[jogador]}=5$.

\subsubsection{Quantidade de Estados Total}

Como demonstrado na equação \ref{eq:estados}, a quantidade de estados existentes para uma partida é o produto de todos os estados. Considerando um caso geral, esse número excede $6\times 10^{37}$, levando mais de $2\times 10^{30}$ anos para calcular todos os estados a uma velocidade de $1 \nicefrac{E_{stado}}{S_{egundo}}$.

\begin{equation} \label{eq:estados} \tag{e.q. Caso Geral}
\begin{split}
	Estados\ &=\ 55^{2}\cdot61^{5}\cdot326\cdot3003^{5}\cdot252^{2}\cdot5\\
	Estados\ &=\ 64'586'224'969'618'554'770'829'159'998'863'764'000\\
	Estados\ &\approx 6\times 10^{37}
\end{split}
\end{equation}

\begin{equation} \label{eq:anos_estados} \tag{e.q. Tempo de Computação dos Estados}
\begin{split}
Anos\ &=\ \dfrac{N_{estados\ distintos}}{\nicefrac{estado}{segundo}\cdot\nicefrac{segundos}{minuto}\cdot\nicefrac{minutos}{hora}\cdot\nicefrac{horas}{dia}\cdot\nicefrac{dias}{ano}}\\
Anos\ &=\ \dfrac{64'586'224'969'618'554'770'829'159'998'863'764'000}{\nicefrac{1}{1}\cdot\nicefrac{60}{1}\cdot\nicefrac{60}{1}\cdot\nicefrac{24}{1}\cdot\nicefrac{365}{1}}\\
Anos\ &=\ 2'048'015'758'803'226'720'862'434'492'416\\
Anos\ &\approx 2\times10^{30}
\end{split}
\end{equation}

\subsection{Podas}

Podas são limites impostos à busca na árvore de possibilidades de jogadas. A seguir estão as possíveis podas descobertas para este problema.

\subsubsection{Poda por Posição do Peão}

No cálculo da quantidade de estados distintos para a posição dos peões foi levado em consideração que um peão poderia estar em qualquer uma de 61 posições. Na verdade, um peão só pode ficar em cima dos discos de sua própria cor, reduzindo esse número para $1+9+5=11$, sendo que o peão pode estar fora do tabuleiro (1), em cima de algum disco de sua cor (9) e em algum degrau da escada (5). Porém, mesmo com essa poda, o número de estados distintos do jogo, de acordo com a equação \ref{eq:poda1}, é superior a $1\times 10^{34}$, demorando mais do que $3\times10^{26}$ anos para resolver um jogo (equação \ref{eq:anos_poda1}).

\begin{equation} \label{eq:poda1} \tag{e.q. Poda por posição}
\begin{split}
	Estados\ &=\ 55^{2}\cdot11^{5}\cdot326\cdot3003^{5}\cdot252^{2}\cdot5\\
	Estados\ &=\ 12'315'559'641'057'482'993'164'099'882'764'000\\
	Estados\ &\approx 1\times 10^{34}
\end{split}
\end{equation}

\begin{equation} \label{eq:anos_poda1} \tag{e.q. Tempo de Computação da Poda 1}
\begin{split}
Anos\ &=\ \dfrac{N_{estados\ distintos}}{\nicefrac{estado}{segundo}\cdot\nicefrac{segundos}{minuto}\cdot\nicefrac{minutos}{hora}\cdot\nicefrac{horas}{dia}\cdot\nicefrac{dias}{ano}}\\
Anos\ &=\ \dfrac{12'315'559'641'057'482'993'164'099'882'764'000}{\nicefrac{1}{1}\cdot\nicefrac{60}{1}\cdot\nicefrac{60}{1}\cdot\nicefrac{24}{1}\cdot\nicefrac{365}{1}}\\
Anos\ &=\ 390'523'834'381'579'220'798'144'512\\
Anos\ &\approx 3\times10^{26}
\end{split}
\end{equation}

\section{Desenvolvimento do Jogo Eletrônico}

Foi desenvolvido um programa para calcular a quantidade aproximada de estados distintos de uma partida reduzida do jogo \emph{Big Points}. Dado $d$ discos e $p$ peões, o programa realiza todos os movimentos possíveis no primeiro turno, movendo uma vez com cada peão, além de capturar o disco atrás e à frente do peão. Então, recursivamente, a função de mover e coletar discos é chamada, cobrindo cada possibilidade de jogada.

O cálculo da quantidade de estados difere do proposto da seção \ref{subsec:quantidade-de-estados} pois não representa o tabuleiro como uma máscara binária, mas sim como um conjunto de caracteres. O caracter 0 representa que aquele disco naquela posição está indisponível, ou seja, já foi capturado por um jogador. Os caracteres 1,2,3,4 e 5 representam as cores Vermelha, Verde, Azul, Amarela e Violeta, respectivamente, e os caracteres \emph{R}, \emph{G}, \emph{B}, \emph{Y}, \emph{V} representam os peões Vermelho, Verde, Azul, Amarelo e Violeta, respectivamente, percorrendo o tabuleiro até o fim. Não foi levado em consideração os discos na mão dos jogadores, e nem os estados da escada.

\lstinputlisting[language=C, firstline=7,label = {cod:boards-state}, caption = {\textit{Quantidade de Estados Aproximados}} ]{../../util/boards_state.cpp }
