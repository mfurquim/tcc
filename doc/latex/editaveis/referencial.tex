\chapter[Fundamentação Teórica]{Fundamentação Teórica}
\label{cha:fundamentacao-teorica}

\section{Histórico da Teoria dos Jogos}
\label{sec:historico-da-teoria-dos-jogos}

Pode-se dizer que a análise de jogos é praticada desde o séculco XVIII
tendo como evidência o trabalho de James Waldergrave ao analisar um jogo
de cartas chamado \emph{Le Her} \cite{Prague_severalmilestones}. No
século seguinte, Augustin Cournot fez uso da teoria dos jogos para
estudos relacionados à política \cite{cournot_1838}. Mais recentemente,
em 1913, Ernst Zermelo publica o primeiro teorema matemático da teoria
dos jogos \cite{zermelo_1913}.

Dois outros grandes matemáticos que se interessaram na teoria dos jogos
foram Émile Borel e John von Neumann. Nas décadas de 1920 e 1930, Emile
Borel publicou três artigos \cite{borel_1921} \cite{borel_1924}
\cite{borel_1927} e um livro \cite{borel_1938} sobre jogos estratégicos,
introduzindo uma noção abstrada sobre jogo estratégico e estratégia
mista. Em 1928, John von Neumann demonstrou que todo jogo finito de soma
zero\footnote{Um jogo soma zero é um
 jogo no qual a vitória de um jogador implica na derrota do outro.} com
duas pessoas possui uma solução em estratégias mistas {[}18{]}. Em 1937,
Neumann forneceu uma nova demonstração baseada em outro teorema (teorema
do ponto fixo de Brouwer). Em 1944, Neumann publicou um trabalho, junto
a Oscar Morgenstern {[}19{]}, e com isso, a teoria dos jogos entrou na
área da economia e matemática aplicada.

Outro matemático que contribuiu para a área foi John Forbes Nash Júnior,
quea publicou quatro artigos importantes para teoria dos jogos
não-cooperativos. Dois destes artigos {[}13, 16{]} provando a existência
de um equilíbrio de estratégias mistas para jogos não-cooperativos,
denominado \textbf{equilíbrio de Nash}, que será explicado na seção
\ref{subsec:solucoes-de-um-jogo}. Nash recebeu o prêmio Nobel em 1994,
junto com John Harsanyi e Reinhard Selten, por suas contribuições para a
teoria dos jogos.
