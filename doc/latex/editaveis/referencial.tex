\chapter[Fundamentação Teórica]{Fundamentação Teórica}
\label{cha:fundamentacao-teorica}

\section{O que é Teoria dos Jogos?}
\label{sec:o-que-e-teoria-dos-jogos}

Teoria dos jogos é o estudo do comportamento estratégico interdependente\footnote{Estratégia interdependente significa que as ações de uma pessoa interfere no resultado da outra, e vice-versa.} \cite{spaniel_2011}, não apenas o estudo de como vencer ou perder em um jogo, apesar de às vezes esses dois fatos coincidem. Isso faz com que o escopo seja mais abranjente, desde comportamentos no qual as duas pessoas devem cooperar para ganhar, ou as duas tentam se ajudar, ou, por fim, comportamento de duas pessoas que tentam vencer individualmente.

\section{Histórico da Teoria dos Jogos}
\label{sec:historico-da-teoria-dos-jogos}

Pode-se dizer que a análise de jogos é praticada desde o séculco XVIII tendo como evidência uma carta escrita por James Waldegrave ao analisar uma versão curta de um jogo de baralho chamado \emph{le Her} \cite[p.~2]{Prague_severalmilestones}, explicado na Seção \ref{subsec:analise-primitiva-do-jogo-le-her}. No século seguinte, Augustin Cournot fez uso da teoria dos jogos para estudos relacionados à política. Mais recentemente, em 1913, Ernst Zermelo publica o primeiro teorema matemático da teoria dos jogos \cite[p.~2]{sartini_IIbienaldasbm}.

Dois grandes matemáticos que se interessaram na teoria dos jogos foram Émile Borel e John von Neumann. Nas décadas de 1920 e 1930, Emile Borel publicou quatro artigos sobre jogos estratégicos \cite[p.~2]{Prague_severalmilestones}, introduzindo uma noção abstrada sobre jogo estratégico e estratégia mista. Em 1928, John von Neumann demonstrou que todo jogo finito de soma zero\footnote{Um jogo soma zero é um jogo no qual a vitória de um jogador implica na derrota do outro.} com duas pessoas possui uma solução em estratégias mistas. Em 1944, Neumann publicou um trabalho junto a Oscar Morgenstern introduzindo a teoria dos jogos na área da economia e matemática aplicada \cite[p.~2--3]{sartini_IIbienaldasbm}.

Outro matemático que contribuiu para a área foi John Forbes Nash Júnior, que publicou quatro artigos importantes para teoria dos jogos não-cooperativos. Dois destes artigos provando a existência de um equilíbrio de \textbf{estratégias mistas} para jogos não-cooperativos, denominado \textbf{equilíbrio de Nash}. Os conceitos de estratégia mista será explicado na Seção \ref{subsubsec:estrategias-mistas} e equilíbrio de nash na Seção \ref{subsubsec:equilibrio-de-nash}. Nash recebeu o prêmio Nobel em 1994, junto com John Harsanyi e Reinhard Selten, por suas contribuições para a teoria dos jogos \cite[p.~3--4]{sartini_IIbienaldasbm}.

\section{Conceitos Fundamentais da Teoria dos Jogos}
\label{sec:conceitos-fundamentais-da-teoria-dos-jogos}

Esta seção introduz os conceitos fundamentais da teoria dos jogos, tais como definição de um jogo não cooperativo, formas de representá-lo e teoremas para encontrar soluções. Para tanto, considere o exemplo do jogo \emph{Renée v Peter} retirado do livro \cite{jones_1980}.

\begin{myex}\label{ex:renee-v-peter}
O jogo \emph{Renée v Peter} é jogado com três cartas: Rei ($K$), Dez ($T$) e Dois ($D$). \emph{Renée} escolhe uma carta dentre as três, coloca-a voltada para baixo e, então, \emph{Peter} escolhe \emph{Alta} ou \emph{Baixa}. \emph{Peter} ganha R\$ 3 se acertar (\emph{Alta} $=\ K$ e \emph{Baixa} $=\ D$), mas deve pagar R\$ 2 a \emph{Renée} se errar.

A terceira possibilidade de jogada de \emph{Renée} é escolher o $T$. Neste caso, se \emph{Peter} tiver escolhido \emph{Baixa}, ele ganha R\$ 2, mas se tiver escolhido \emph{alta}, \emph{Renée} deve escolher outra carta (dentre as duas restantes). Se \emph{Peter} acertar a segunda carta, \emph{Renée} deve pagar R\$ 1, mas ganha R\$ 3 se \emph{Peter} errar.
\end{myex}

\subsection{Definição de Jogo Não Cooperativo}
\label{subsec:definicao-de-jogo-nao-cooperativo}

Tendo a definição de um jogo como sendo uma atividade interativa e competitiva no qual os jogadores devem obedecer a um determinado conjunto de regras, um \textbf{jogo não cooperativo} é um jogo que não permite nenhum tipo de acordo entre os jogadores e o ganho de cada jogador é determinado pelo conjunto de regras \cite{jones_1980}.

\subsection{Jogo Simétrico e Assimétrico}
\label{subsec:jogo-simetrico-e-assimetrico}

Um \textbf{jogo simétrico} é um jogo no qual as regras são as mesmas para ambos os jogadores. Em termos mais matemáticos, tem-se a Definição \ref{def:jogo-simetrico}.

\begin{mydef}\label{def:jogo-simetrico}
Dado uma matriz $G$ de um jogo, $G$ é a representação de um \textbf{jogo simétrico} se for uma matriz quadrada e $g_{ij} = -g_{ji}$ para cada par $i$, $j$. Em particular, $g_{ii} = 0$ para todo $i$. Portanto, $G$ é uma representação de jogo simétrica se for uma matriz simétrica simpléctica\footnote{Uma matriz é simpléctica se sua diagonal principal for 0.}\footnote{Do inglês \emph{if its matrix is skew-symmetric}.}. \cite{jones_1980}
\end{mydef}

\subsection{Função Utilidade (\emph{payoff})}
\label{subsec:funcao-utilidade}

Uma \textbf{função utilidade} para um jogador determina o ganho para aquele jogador ao executar sua estratégia. Uma definição mais matemática é dada por Sartini et al como:

\begin{mydef}\label{def:funcao-utilidade}
Uma \emph{função utilidade} para o jogador $i$ é uma função real $u_i:\Psi_T \rightarrow \mathbb{R}$ definida no conjunto $\Psi_T$ das jogadas completas. O valor de $u_i$ em cada jogada determina o \emph{payoff} do jogador $i$ para esta jogada. \cite{sartini_IIbienaldasbm}
\end{mydef}

\subsection{Jogo de dois jogadores e soma zero}
\label{subsec:jogo-de-dois-jogadores-e-soma-zero}

Um jogo de dois jogadores e de soma zero significa que a vitória de um jogador implica na derrota do outro. Por convenção, o ganho ao final do jogo é representado pelo ganho do primeiro jogador, normalmente seguindo a seguinte representação: $1$ caso ele ganhe, $-1$ caso ele perca e $0$ caso ele empate. Caso explicado nas regras do jogo, o ganho pode ser calculado pela \textbf{função utilidade}.

\subsection{Forma Extensiva}
\label{subsec:forma-extensiva}

Uma das formas de representar um jogo, de tal maneira que seja possível analisá-lo em seguida, é a \textbf{forma extensiva} faz uso de uma estrutura de árvore, onde para representar os estados do jogo se faz uso dos nós da árvore, as jogadas possíveis a partir daquele estado são as arestas, e o ganho para o primeiro jogador são as folhas. A Figura \ref{fig:renee-v-peter} representa o Exemplo \ref{ex:renee-v-peter} na forma extensiva.

\begin{figure}[ht]
	\centering
	\begin{tikzpicture}
	  [
	    grow                    = down,
	    edge from parent/.style = {draw, -latex},
	    every node/.style       = {font=\footnotesize},
	    sloped,
		level 1/.style			= {
			sibling distance=2.5cm,
			level distance=1.5cm
		},
		level 2/.style			= {
			sibling distance=1.5cm,
			level distance=2cm
		},
		level 3/.style			= {
			sibling distance=2cm,
			level distance=1.5cm
		},
		level 4/.style			= {
			sibling distance=1.5cm,
			level distance=2cm
		}
		]
		\node {\emph{Renée}}
		child {
			node (P1){\emph{Peter}} {
				child {
					node {-3}
					edge from parent node [above] {\emph{Alta}}
				}
				child {
					node {2}
					edge from parent node [above] {\emph{Baixa}}
				}
			}
			edge from parent node [above] {$K$}
		}
		child {
			node (P2) {\emph{Peter}} {
				child {
					node (R){\emph{Renée}} {
						child {
							node (P21) {\emph{Peter}} {
							child {
								node {-1}
								edge from parent node [above] {\emph{Alta}}
							}
							child {
								node {3}
								edge from parent node [above] {\emph{Baixa}}
							}
							}
							edge from parent node [above] {$K$}
						}
						child {
							node (P22) {\emph{Peter}} {
							child {
								node {3}
								edge from parent node [above] {\emph{Alta}}
							}
							child {
								node {-1}
								edge from parent node [above] {\emph{Baixa}}
							}
							}
							edge from parent node [above] {$D$}
						}
					}
					edge from parent node [above] {\emph{Alta}}
				}
				child {
				node {1}
				edge from parent node [above] {\emph{Baixa}}
				}
			}
			edge from parent node [above] {$T$}
		}
		child {
			node (P3) {\emph{Peter}} {
				child {
					node {2}
					edge from parent node [above] {\emph{Alta}}
				}
				child {
					node {-3}
					edge from parent node [above] {\emph{Baixa}}
				}
			}
			edge from parent node [above] {$D$}
		};
		\draw[thick, rounded corners] ($(P1.north west)+(-0.1,-0.1)$) rectangle ($(P3.south east)+(0.1,+0.1)$);
		\draw[thick, rounded corners] ($(P21.north west)+(-0.1,-0.1)$) rectangle ($(P22.south east)+(0.1,+0.1)$);
	\end{tikzpicture}
	\caption{Árvore do jogo Renée v Peter}
	\label{fig:renee-v-peter}
\end{figure}

\subsection{Conjunto Informação}
\label{subsec:conjunto-informacao}

Um \textbf{conjunto informação}, de acordo com Sartini et al, é um conjunto de estados do jogo que pertencem ao mesmo jogador e os movimentos disponíveis são os mesmos para cada estado \cite{sartini_IIbienaldasbm}. Jones diz, ainda, que um termo mais apropriado seria \textbf{conjunto de ignorância} \cite{jones_1980}, pois indica que ao fazer um movimento, o jogador sabe em qual conjunto informação ele se encontra, mas não sabe dizer em qual estado está. Se o jogo está sendo representado por uma árvore, como na Figura \ref{fig:renee-v-peter}, então o conjunto de informação é representado pelo

\subsection{Estratégia Pura}
\label{subsec:estrategia-pura}

\textbf{Estratégia pura} é um conjunto de escolhas a se fazer para cada momento de decisão possível. Voltando ao Exemplo \ref{ex:renee-v-peter}, \emph{Peter} pode escolher \emph{Baixa} em sua primeira jogada, ganhando ganhando duas das vezes e perdendo uma, ou pode escolher \emph{Alta} com uma possibilidade de ter que tomar outra decisão. Considerando essas possibilidades, tem-se que as estratégias puras de \emph{Peter} são:

\begin{enumerate}
	\tightlist
	\item[$ PI\ -$] Escolher \emph{Alta}; Se \emph{Renée} escolher $T$, escolher \emph{Alta}.
	\item[$ PII\ -$] Escolher \emph{Alta}; Se \emph{Renée} escolher $T$, escolher \emph{Baixa}.
	\item[$ PIII\ -$] Escolher \emph{Baixa}.
\end{enumerate}

Enquanto para \emph{Renée}, suas opções são escolher entre as três cartas, mas se escolher $T$ e tiver que escolher outra carta, deve escolher uma entre as duas restantes. Com isso, tem-se as seguintes estratégias puras de \emph{Renée}:

\begin{enumerate}
	\tightlist
	\item[$ RI\ -$] Escolher \emph{K}.
	\item[$ RII\ -$] Escolher \emph{T}; Se \emph{Peter} escolher \emph{Alta}, escolher \emph{K}.
	\item[$ RIII\ -$] Escolher \emph{T}; Se \emph{Peter} escolher \emph{Alta}, escolher \emph{D}.
	\item[$ RIV\ -$] Escolher \emph{D}.
\end{enumerate}

\subsection{Forma Normal}
\label{subsec:forma-normal}

Essa propriedade \textbf{estratégias puras} permite uma outra representação de um jogo chamado de \textbf{forma normal}. Nesta forma o jogo é representado por uma matriz, onde na linha tem-se as estratégias puras do primeiro jogador e na coluna, as do segundo jogador. No caso do Exemplo \ref{ex:renee-v-peter}, o jogo é representado pela Tabela \ref{tab:forma-normal-do-jogo-renee-v-peter}.

\begin{table}[ht]
\centering
\begin{tabular}{cc|ccc}
\hline
 &  & \multicolumn{3}{c}{\emph{Peter}}\tabularnewline
 &  & I & II & III\tabularnewline
\hline
\multirow{4}{*}{\begin{turn}{90}
\emph{Renée}
\end{turn}} & I & \emph{P} & \emph{P} & \emph{R}\tabularnewline
 & II & \emph{P} & \emph{R} & \emph{P}\tabularnewline
 & III & \emph{R} & \emph{P} & \emph{P}\tabularnewline
 & IV & \emph{R} & \emph{R} & \emph{P}\tabularnewline
\hline
\end{tabular}
\caption{Forma normal do jogo \emph{Renée v Peter}}
\label{tab:forma-normal-do-jogo-renee-v-peter}
\end{table}

\subsection{Matriz de \emph{payoff}}
\label{subsec:matriz-de-payoff}

Uma matriz de \emph{payoff} nada mais é do que a representação em forma normal com o ganho indicado em cada situação. No caso do Exemplo \ref{ex:renee-v-peter} o ganho é e, com isso, tem-se a Tabela \ref{tab:matriz-de-payoff-do-jogo-renee-v-peter}.

\begin{table}[ht]
\centering
\begin{tabular}{cc|ccc}
\hline
 &  & \multicolumn{3}{c}{\emph{Peter}}\tabularnewline
 &  & I & II & III\tabularnewline
\hline
\multirow{4}{*}{\begin{turn}{90}
\emph{Renée}
\end{turn}} & I & -3 & -3 & 2\tabularnewline
 & II & -1 & 3 & -2\tabularnewline
 & III & 3 & -1 & -2\tabularnewline
 & IV & 2 & 2 & -3\tabularnewline
\hline
\end{tabular}
\caption{Matriz de \emph{payoff} do jogo \emph{Renée v Peter}, fonte: \cite{spaniel_2011}}
\label{tab:matriz-de-payoff-do-jogo-renee-v-peter}
\end{table}

\subsection{Soluções para jogos}
Uma solução de um jogo é uma prescrição ou previsão sobre o resultado do jogo. Há soluções para os jogos que as escolhas dos jogadores são feitas simultaneamente e soluções para os jogos que possuem turnos. Para o primeiro caso, as soluções são encontradas fazendo uso de \textbf{dominância estrita iterada}, teorema \textbf{minimax} ou \textbf{equilíbrio de Nash}, podendo ser com \textbf{estratégias mistas} ou não, enquanto para o segundo caso a solução é encontrada utilizando-se de \textbf{\emph{backward induction}}.

\subsubsection{Dominância estrita}

É dito que uma estratégia é {\bfseries estritamente dominada} para um jogador se esta estratégia gera um ganho maior do que qualquer outra estratégia independente do que o outro jogador fizer. Considera-se que jogadores racionais nunca fazem uso de estratégias estritamente dominadas, pois não há motivos para escolher uma estratégia que sempre será pior em todos os casos. O exemplo do {\bfseries dilema do prisioneiro} demonstra tal conceito.

Formulado por Albert W. Tucker em 1950 \cite{sartini_IIbienaldasbm}, o dilema do prisioneiro é, provavelmente, um dos exemplos mais conhecidos na teoria dos jogos. O propósito de Tucker foi ilustrar a dificuldade de se analisar certos tipos de jogos por razões que ficarão óbvias após o exemplo.


\begin{myex}\label{ex:dilema-do-prisioneiro}
Dois suspeitos são presos com suspeita de roubo mas os policiais podem apenas provar que os suspeitos invadiram o local. Precisando da confissão dos criminosos, o policial faz a seguinte proposta:
\begin{itemize}
	\tightlist
	\item Se nenhum confessar o roubo, o policial vai prendê-los por intrusão.
	\item Se um confessar e o outro não, o que confessou será liberto e o calado será preso por 12 meses.
	\item Se os dois confessarem, ambos serão presos por 8 meses.
\end{itemize}
Considerando que os dois suspeitos desejam minimizar seu tempo na cadeia, eles devem confessar à polícia?
\end{myex}

Dado essas informações do Exemplo \ref{ex:dilema-do-prisioneiro}, é possível representá-las no formato {\bfseries matriz de \emph{payoffs}} de acordo com a Tabela \ref{tab:dilema-prisioneiro}, onde o ganho do \emph{jogador linha} é representado pelo número à esquerda dentro de uma célula, e o ganho do \emph{jogador coluna} é representado à direita.

\begin{table}[ht]
\centering
\begin{tabular}{|c|c|c|c|}
	\cline{3-4}
	\multicolumn{1}{c}{} &  & \multicolumn{2}{c|}{{\bfseries Coluna}}\tabularnewline
	\cline{3-4}
	\multicolumn{1}{c}{} &  & {\scshape Negar}\  & {\scshape Confessar}\ \tabularnewline
	\hline
	\multirow{2}{*}{\begin{turn}{90}
	{\bfseries Linha}
	\end{turn}} & \begin{turn}{90}
	{\scshape Negar}\
	\end{turn} & {\Large(}{\Large -1,}{\Large -1)} & {\Large(}{\Large -12,}{\Large 0)}\tabularnewline
	\cline{2-4}
	 & \begin{turn}{90}
	{\scshape Confessar}\
	\end{turn} & {\Large(}{\Large 0,}{\Large -12)} & {\Large(}{\Large -8,}{\Large -8)}\tabularnewline
	\hline
\end{tabular}
\caption{Dilema do prisioneiro, fonte: \cite{spaniel_2011}}
\label{tab:dilema-prisioneiro}
\end{table}

Para resolver este jogo é preciso raciocinar como um jogador responderia de acordo com a ação do outro. Supondo que o \emph{jogador coluna} fique \emph{Negar}, o \emph{jogador linha} pode \emph{Negar} e ir pra cadeia por 1 mês, sendo representado na Tabela \ref{tab:dilema-prisioneiro} pela célula superior esquerda, ou aceitar a proposta do policial e \emph{Confessar} o crime que iriam cometer, sendo representado pela célula inferior esquerda. Como os jogadores querem minimizar seu tempo na prisão, que é representado por um valor negativo, deve-se buscar o maior valor dentre essas duas escolhas, que neste caso é \emph{Confessar} com um ganho de \emph{0}. Observando a outra possibilidade do \emph{jogador coluna}, que seria \emph{Confessar}, o \emph{jogador linha} teria um ganho de \emph{-12} ao \emph{Negar} e um ganho de \emph{-8} ao \emph{Confessar}. Em ambos os casos, o \emph{jogador linha} terá um melhor ganho ao \emph{Confessar}, e como o jogo é simétrico o mesmo raciocínio pode ser feito para o \emph{jogador coluna}.

Essa preferência de \emph{Confessar} a \emph{Negar} para cada escolha do outro jogador (\emph{Negar} ou \emph{Confessar}) é dito que \emph{Confessar} é estritamente dominante.

\subsubsection{Eliminação iterada de estratégias estritamente dominadas}

Eliminação por dominância estrita iterada, como o próprio nome já diz, é um processo iterado no qual estratégias são eliminadas por serem dominadas estritamente \cite{spaniel_2011}. Se uma estratégia for estritamente dominada, elimine-a imediatamente, não importando a ordem da eliminação. Se ao final do processo sobrar apenas uma única célula, este resultado será alcançado começando a eliminação por qualquer estratégia estritamente dominada. Considere a Tabela \ref{tab:dominancia-estrita-iterada}.

\begin{table}[ht]
\centering
\begin{tabular}{|c|c|c|c|c|}
\cline{3-5}
\multicolumn{1}{c}{} &  & \multicolumn{3}{c|}{\textbf{Jogador Coluna}}\tabularnewline
\cline{3-5}
\multicolumn{1}{c}{} &  & \textsc{Esquerda} & \textsc{Centro} & \textsc{Direita}\tabularnewline
\hline
\multirow{3}{*}{\begin{turn}{90}
\textbf{Jogador Linha}
\end{turn}} & \begin{turn}{90}
\textsc{Cima}
\end{turn} & {\Large(13,3)} & {\Large(1,4)} & {\Large(7,3)} \tabularnewline
\cline{2-5}
 & \begin{turn}{90}
\textsc{Meio}
\end{turn} & {\Large(4,1)} & {\Large(3,3)} & {\Large(6,2)} \tabularnewline
\cline{2-5}
 & \begin{turn}{90}
\textsc{Baixo}
\end{turn} &  {\Large(-1,9)} & {\Large(2,8)} & {\Large(8,-1)} \tabularnewline
\hline
\end{tabular}
\caption{Exemplo de dominância estrita iterada, fonte: \cite{spaniel_2011}}
\label{tab:dominancia-estrita-iterada}
\end{table}

Na Tabela \ref{tab:dominancia-estrita-iterada}, o \emph{jogador linha} tem três estratégias \emph{cima}, \emph{meio} e \emph{baixo}, enquanto o \emph{jogador coluna} possui as estratégias \emph{esquerda}, \emph{centro} e \emph{direita}, gerando um total de 9 resultados.
Observando as estratégias do \emph{jogador linha}, não é possível fazer nenhuma dominância estrita, pois o \emph{jogador linha} possui uma preferência diferente de suas próprias estratégias para cada estratégia do \emph{jogador coluna}. No caso da estratégia \emph{esquerda}, a melhor estratégia para o \emph{jogador linha} é \emph{cima}, assim como \emph{centro} e \emph{meio}, e \emph{direita} e \emph{baixo}.

Porém, observando as estratégias \emph{centro} e \emph{direita}, é possível eliminar a segunda estratégia por dominância estrita, pois, para o \emph{jogador coluna}, todos os ganhos da estratégia \emph{centro} são melhores do que os ganhos da estratégia \emph{direita}. Com a eliminação da estratégia \emph{direita}, é possível eliminar a estratégia \emph{baixo}, que é estritamente dominada pela estratégia \emph{meio} para o \emph{jogador linha}. Com essas duas eliminações, é possível eliminar as estratégias \emph{esquerda}, pois, para o jogador \emph{jogador coluna}, é estritamente dominada pela estratégia \emph{centro}, e por fim, para o jogador \emph{jogador linha}, a estratégia \emph{cima} é estritamente dominada pela estratégia \emph{meio}.

No final da eliminação iterada de estratégias estritamente dominadas. sobra apenas uma única célula da Tabela \ref{tab:dominancia-estrita-iterada}, como demonstrado na Tabela \ref{tab:final-dominancia-estrita-iterada}, e é chamado de \textbf{equilíbrio de estratégia dominante}.

\begin{table}[ht]
\centering
\begin{tabular}{|c|c|}
\cline{2-2}
\multicolumn{1}{c|}{} & {\scshape Centro} \tabularnewline
\hline
\begin{turn}{90}
{\scshape Meio}
\end{turn} &  {\Large(}{\Large 3,}{\Large 3)}\tabularnewline
\hline
\end{tabular}
\caption{Final do exemplo de dominância estrita iterada, fonte: \cite{spaniel_2011}}
\label{tab:final-dominancia-estrita-iterada}
\end{table}

\subsubsection{Estratégias mistas}

Considere a situação a seguir: Dois caçadores devem decidir o que caçar no dia e levar o equipamento apropriado. Eles sabem que, no local de caça, existem duas lebres, que valem uma unidade de carne cada, e um veado, que vale seis unidades de carne. O veado vale mais carne dividindo para cada um do que a soma das duas lebres, mas é preciso o auxílio do outro caçador para caçar um veado, enquanto as lebres podem ser caçadas sem nenhuma ajuda \cite{spaniel_2011}. Estas informações são condensadas na Tabela \ref{tab:caca-ao-viado}.

\begin{table}[ht]
\centering
\begin{tabular}{|c|c|c|c|}
	\cline{3-4}
	\multicolumn{1}{c}{} &  & \multicolumn{2}{c|}{{\bfseries Coluna}}\tabularnewline
	\cline{3-4}
	\multicolumn{1}{c}{} &  & {\scshape Veado}\  & {\scshape Lebre}\ \tabularnewline
	\hline
	\multirow{2}{*}{\begin{turn}{90}
	{\bfseries Linha}
	\end{turn}} & \begin{turn}{90}
	{\scshape Veado}\
	\end{turn} & {\Large(3,3)} & {\Large(0,2)}\tabularnewline
	\cline{2-4}
	 & \begin{turn}{90}
	{\scshape Lebre}\
	\end{turn} & {\Large(2,0)} & {\Large(1,1)}\tabularnewline
	\hline
\end{tabular}
\caption{Caça ao Veado, fonte: \cite{spaniel_2011}}
\label{tab:caca-ao-viado}
\end{table}

Pode parecer óbvio que o resultado desse jogo é os dois caçadores escolherem caçar \emph{Veado}, pois o ganho para ambos é maior do que em qualquer outra situação, mas esse não é bem o caso.

\subsubsection{Teorema Minimax}
\label{subsubsec:teorema-minimax}

\subsubsection{Equilíbrio de Nash}
\label{subsubsec:equilibrio-de-nash}

\subsubsection{Estratégias Mistas}
\label{subsubsec:estrategias-mistas}


\subsection{Análise primitiva do jogo \emph{le Her}}
\label{subsec:analise-primitiva-do-jogo-le-her}

O objetivo do jogo \emph{le Her} é terminar o jogo com a carta mais alta, sendo que o baralho é contado de Ás ($A$) à Rei ($K$). Essa versão reduzida podia ser jogada apenas com dois jogadores, um deles chamado \emph{dealer} e outro \emph{receiver}. O \emph{dealer} embaralha as cartas e distribui uma carta para o \emph{receiver} e uma para si. O \emph{receiver} tem a escolha de manter sua carta ou trocá-la com o \emph{dealer}, e em seguida o \emph{dealer} tem a mesma opção de manter ou de trocar sua carta com uma carta nova do baralho. A única regra que impede a troca é o caso da carta recebida ser um Rei ($K$), neste caso a troca deve ser desfeita e o jogador mantém sua carta original.

%Vários artigos apontam

\section{O Jogo \emph{Big Points}}
\subsection{Conceito do Jogo}
\label{subsec:conceito-do-jogo}
\emph{Big Points} é um jogo abstrato e estratégico com uma mecânica de colecionar peças. São cinco peões de cores distintas, que podem ser usadas por qualquer jogador, para percorrer um caminho de discos coloridos até chegar ao pódio. Durante o percurso, os jogadores coletam alguns destes discos. A pontuação de cada jogador é determinada a partir da ordem de chegada dos peões ao pódio e a quantidade de discos adquiridos de cada cor. Ganha o jogador com a maior pontuação.

\subsection{Regras do Jogo}
\label{subsec:regras-do-jogo}
O jogo \emph{Big Points} pode ser jogado de dois a cinco jogadores. No seu turno, o jogador escolhe qualquer um dos cinco peões, que possuem as cores vermelha, verde, azul, amarela, violeta e preta, e o move para cima do próximo disco de sua mesma cor. Em seguida, o jogador deve pegar o próximo disco disponível\footnote{É dito indisponível aqueles discos que já foram pegos por algum jogador ou que possuem um peão em cima.} à frente ou atrás deste peão. Caso não haja discos atrás (à frente) do peão, o jogador deve pegar o disco que está à frente (atrás). Existem sete cores de discos, sendo elas as cores branca, preta e as cinco cores dos peões. Caso o jogador já possua um disco preto no começo do seu turno, tal jogador pode escolher descartá-lo após sua jogada para realizar um segundo movimento. Este movimento pode ser com qualquer cor de peão, como uma jogada normal, mas dessa vez o movimento pode ser feito para trás, se houver discos de sua cor para mover.

Se o jogador escolher um peão o qual não possua discos de sua cor à frente, então este peão deve ir para a posição mais alta da escada, posicionada ao final da trilha de discos, e o jogador pega um disco daquela mesma cor, reservado ao lado da escada. Uma vez em cima da escada, nenhum jogador pode escolher aquele peão para movimentá-lo. O jogo termina quando todos os peões estiverem em cima da escada e, consequente, nenhuma jogada possível restante.

A pontuação de cada jogador é contada de acordo com a posição dos peões na escada e a quantidade de discos de suas cores na mão do jogador. Os discos da cor do peão no topo da escada valem 4, os discos da cor do peão na posição anterior valem 3, e assim sucessivamente até os discos da cor do peão na última posição que não valem ponto algum. O disco branco vale um ponto para cada cor de disco diferente que o jogador possue em sua mão, com a exceção da cor branca\footnote{A pontuação máxima de um disco branco é igual a 6. Um ponto para cada uma das seguintes cores de disco: Vermelho, Verde, Azul, Amarelo, Violeta e Preto.}.
