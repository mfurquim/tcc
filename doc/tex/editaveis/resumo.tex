\begin{resumo}

A Teoria dos Jogos estuda as melhores estratégias dos jogadores em uma determinada situação de conflito.
Este trabalho faz uso do teorema \emph{minimax} para solucionar versões reduzidas do jogo \emph{Big Points} com o propósito de investigar o balanceamento do jogo.
O jogo foi reduzido em relação ao tipo e quantidade de certas peças, pois solucionar o jogo completo exigiria um trabalho computacional imenso.
Utilizando-se técnicas de memorização, são implementadas duas funções para separar a lógica do jogo da programação dinâmica.
Os resultados após a escrita do código, e a execução do programa, sugere que o jogo de \emph{Big Points} completo seja desbalanceado.

 \vspace{\onelineskip}

 \noindent
 \textbf{Palavras-chaves}: Teoria dos Jogos, Análise Computacional dos Jogos.
\end{resumo}
