\renewcommand{\backrefpagesname}{Citado na(s) página(s):~}
\newcommand{\nocontentsline}[3]{}
\newcommand{\tocless}[2]{\bgroup\let\addcontentsline=\nocontentsline#1{#2}\egroup}
\renewcommand*{\lstlistingname}{Código}
\renewcommand*{\lstlistlistingname}{Lista de \lstlistingname s}
\renewcommand{\backref}{}
\renewcommand*{\backrefalt}[4]{
	\ifcase #1 %
		Nenhuma citação no texto.%
	\or
		Citado na página #2.%
	\else
		Citado #1 vezes nas páginas #2.%
	\fi}%
% ---

\providecommand{\tightlist}{%
	  \setlength{\itemsep}{0pt}\setlength{\parskip}{0pt}}

\lstset{%
language=C++,           %linguagem
numbers=left,           %posição dos números
stepnumber=1,           %frequencia de aparição dos números
numbersep=5pt,
%numberstyle=\zebra{gray!15}{white!35},
%basewidth={0.6em,0.45em},
%fontadjust=true,
%mathescape=true,
tabsize=4,
commentstyle=\color{blue},
literate={á}{{\'a}}1 {à}{{\`a}}1 {ã}{{\~a}}1 {é}{{\'e}}1 {É}{{\'E}}1 {ê}{{\^e}}1 {õ}{{\~o}}1 {í}{{\'i}}1 {ó}{{\'o}}1 {ú}{{\'u}}1 {ç}{{\c c}}1 {³}{{$^3$}}1 {Ω}{{$\Omega$}}1
%breaklines=true,
%showstringspaces=false,
%stringstyle=\color{cyan},
%basicstyle=\small\ttfamily
}
\usetikzlibrary{shapes,snakes}
\usetikzlibrary{calc}

\theoremstyle{definition}
\newtheorem{myex}{Exemplo}
\newtheorem{mydef}{Definição}
