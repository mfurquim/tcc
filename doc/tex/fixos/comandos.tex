\renewcommand{\backrefpagesname}{Citado na(s) página(s):~}
\newcommand{\nocontentsline}[3]{}
\newcommand{\tocless}[2]{\bgroup\let\addcontentsline=\nocontentsline#1{#2}\egroup}
\renewcommand*{\lstlistingname}{Código}
\renewcommand*{\lstlistlistingname}{Lista de Códigos}
\renewcommand{\backref}{}
\renewcommand*{\backrefalt}[4]{
	\ifcase #1 %
		Nenhuma citação no texto.%
	\or
		Citado na página #2.%
	\else
		Citado #1 vezes nas páginas #2.%
	\fi}%
% ---
\counterwithout{equation}{section}

\providecommand{\tightlist}{%
	  \setlength{\itemsep}{0pt}\setlength{\parskip}{0pt}}

\definecolor{mygreen}{rgb}{0,0.6,0}
\definecolor{mygray}{rgb}{0.5,0.5,0.5}
\definecolor{mymauve}{rgb}{0.58,0,0.82}
\lstloadlanguages{[GNU]C++} % Load Cpp syntax for listings, for a list of other languages supported see: ftp://ftp.tex.ac.uk/tex-archive/macros/latex/contrib/listings/listings.pdf

%\lstset{
%    backgroundcolor=\color{white},   % choose the background color; you must add \usepackage{color} or \usepackage{xcolor}
%    basicstyle=\footnotesize,        % the size of the fonts that are used for the code
%    breakatwhitespace=false,         % sets if automatic breaks should only happen at whitespace
%    breaklines=true,                 % sets automatic line breaking
%    captionpos=b,                    % sets the caption-position to bottom
%    commentstyle=\color{mygreen},    % comment style
%    deletekeywords={...},            % if you want to delete keywords from the given language
%    escapeinside={\%*}{*)},          % if you want to add LaTeX within your code
%    extendedchars=true,              % lets you use non-ASCII characters; for 8-bits encodings only, does not work with UTF-8
%    frame=single,                    % adds a frame around the code
%    keepspaces=true,                 % keeps spaces in text, useful for keeping indentation of code (possibly needs columns=flexible)
%    keywordstyle=\color{blue},       % keyword style
%    language=[GNU]C++,                 % the language of the code
%    otherkeywords={*,...},           % if you want to add more keywords to the set
%    numbers=left,                    % where to put the line-numbers; possible values are (none, left, right)
%    numbersep=5pt,                   % how far the line-numbers are from the code
%    numberstyle=\tiny\color{mygray}, % the style that is used for the line-numbers
%    rulecolor=\color{black},         % if not set, the frame-color may be changed on line-breaks within not-black text (e.g. comments (green here))
%    showspaces=false,                % show spaces everywhere adding particular underscores; it overrides 'showstringspaces'
%    showstringspaces=false,          % underline spaces within strings only
%    showtabs=false,                  % show tabs within strings adding particular underscores
%    stepnumber=2,                    % the step between two line-numbers. If it's 1, each line will be numbered
%    stringstyle=\color{mymauve},     % string literal style
%    tabsize=2,                       % sets default tabsize to 2 spaces
%literate={á}{{\'a}}1 {à}{{\`a}}1 {ã}{{\~a}}1 {é}{{\'e}}1 {É}{{\'E}}1 {ê}{{\^e}}1 {õ}{{\~o}}1 {í}{{\'i}}1 {ó}{{\'o}}1 {ú}{{\'u}}1 {ç}{{\c c}}1 {³}{{$^3$}}1 {Ω}{{$\Omega$}}1,
%    title=\lstname                   % show the filename of files included with \lstinputlisting; also try caption instead of title
%}

\newcommand\indexkeywords[1]{\index{keywords, #1}}

\lstset{%
	language=C++,           %linguagem
    showspaces=false,                % show spaces everywhere adding particular underscores; it overrides 'showstringspaces'
    showstringspaces=false,          % underline spaces within strings only
    showtabs=false,                  % show tabs within strings adding particular underscores
    stringstyle=\color{mymauve},     % string literal style
	numbers=left,           %posição dos números
	stepnumber=1,           %frequencia de aparição dos números
    backgroundcolor=\color{white},   % choose the background color; you must add \usepackage{color} or \usepackage{xcolor}
    breakatwhitespace=true,         % sets if automatic breaks should only happen at whitespace
    breaklines=true,                 % sets automatic line breaking
    captionpos=t,                    % sets the caption-position to top
    commentstyle=\color{mygreen},    % comment style
    deletekeywords={...},            % if you want to delete keywords from the given language
    escapeinside={\%*}{*)},          % if you want to add LaTeX within your code
    extendedchars=true,              % lets you use non-ASCII characters; for 8-bits encodings only, does not work with UTF-8
	%numberstyle=\zebra{gray!15}{white!35},
	%basewidth={0.6em,0.45em},
	fontadjust=true,
	%mathescape=true,
	tabsize=4,
	literate={á}{{\'a}}1 {à}{{\`a}}1 {ã}{{\~a}}1 {é}{{\'e}}1 {É}{{\'E}}1 {ê}{{\^e}}1 {õ}{{\~o}}1 {í}{{\'i}}1 {ó}{{\'o}}1 {ú}{{\'u}}1 {ç}{{\c c}}1 {³}{{$^3$}}1 {Ω}{{$\Omega$}}1,
    keepspaces=true,                 % keeps spaces in text, useful for keeping indentation of code (possibly needs columns=flexible)
    keywordstyle=\color{blue},       % keyword style
    language=[GNU]C++,                 % the language of the code
    otherkeywords={*,...},           % if you want to add more keywords to the set
    numbers=left,                    % where to put the line-numbers; possible values are (none, left, right)
    numbersep=3pt,                   % how far the line-numbers are from the code
    numberstyle=\tiny\color{mygray}, % the style that is used for the line-numbers
	numberbychapter=false,
	procnamekeys={program, procedure, function, código},
	indexprocnames=true,
	indexstyle=\index[keywords],
	index={square},
	index={[2]root},
	%showstringspaces=false,
	%stringstyle=\color{cyan},
	basicstyle=\scriptsize        % the size of the fonts that are used for the code
}
\usetikzlibrary{shapes,snakes}
\usetikzlibrary{calc}

\theoremstyle{definition}
\newtheorem{myex}{Exemplo}
\newtheorem{mydef}{Definição}
