\section{Resumo}\label{resumo}

A área de Teoria dos Jogos estuda as melhores estratégias dos jogadores
em um determinado jogo. Aplicando suas teorias em um jogo de tabuleiro
eletrônico, este trabalho propõe analisar o jogo \emph{Big Points} a
partir de um determinado estado e, como resultado, identificar as
melhores heurísticas para os jogadores e uma possível inteligência
artificial.

\section{Introdução}\label{introduuxe7uxe3o}

∑ ∀

\begin{itemize}
\itemsep1pt\parskip0pt\parsep0pt
\item
  História da teoria dos jogos
\item
  Definição de teoria dos jogos
\item
  Jogo a ser analisado
\item
  \emph{Regra do jogo}
\end{itemize}

\section{Materiais e Métodos}\label{materiais-e-muxe9todos}

\begin{itemize}
\itemsep1pt\parskip0pt\parsep0pt
\item
  O jogo eletrônico está sendo implementado
\item
  Foi feito uma análise combinatória inicial para descobrir a
  possibilidade de computar todas as possíveis jogadas de um ou de
  vários jogos. Chegou-se na conclusão que não era factível
\end{itemize}

\subsection{Análise combinatória}\label{anuxe1lise-combinatuxf3ria}

O jobo \emph{Big Points} possui cinco peões de cores distintas, pode ser
jogado de dois a cinco jogadores, e dos 55 discos totais, cinco são
brancos, cinco são pretos e nove de cada um das cinco cores restantes.
Os peões podem estar em cima de um disco ou em uma das cinco posições da
escada.

Cada partida diferente pode ser representado pela quantidade de
jogadores e pela posição inicial dos discos, compondo o tabuleiro. Desta
forma temos que o número de jogos distintos é $10^{41}$ \textless{}
\#Jogos Distintos \textless{} $10^{42}$. Caso fosse possível exaurir
todas as possibilidades de cada jogo em um segundo, o tempo necessário
para fazer isso para todos os jogos distintos estaria em torno de
$10^{35}$ anos.

A quantidade de memória necessária para armazenar um \emph{estado} do
jogo depende de quais características descrevem um \emph{estado}. Como
dito anteriormente, o jogo pode ser jogado até cinco jogadores, possui
um tabuleiro com 55 discos, uma escada com cinco degraus, e cinco peões
no qual a posição varia entre 0 e 60. Considerando as estruturas
\emph{array} Feito um cálculo simples de soma das estruturas utilizadas,
o valor, em bytes, para armazenar um \emph{estado} é 47 bytes.

A quantidade de estados distintos para cada partida é calculado da
sequinte maneira: cada um dos cinco possíveis jogadores pode ter entre
zero e cinco discos das cores branco e preta, assim como pode ter entre
zero e dez discos das cores restantes; cada peão pode estar em uma
posição entre zero e dez (considerando apenas os de sua cor); e, por
fim, cada espaço de disco no tabuleiro pode ou não estar ocupado.
Partindo destas informações, temos o cálculo \ref{eq_poda1}

\begin{itemize}
\itemsep1pt\parskip0pt\parsep0pt
\item
  \textbf{Poda 1} A posição dos peões pode ser apenas de sua respectiva
  cor. Com essa poda, o número de estados distintos de um jogo é
  reduzido, mas ainda se encontra na ordem de $10^{21}$.
\end{itemize}

\section{Resultados}\label{resultados}

?

\section{Discussão e Conclusões}\label{discussuxe3o-e-conclusuxf5es}

?

\section{Apêndice}\label{apuxeandice}

\subsection{Cálculos}\label{cuxe1lculos}

\begin{equation} \label{eq_partidas} \tag{e.q. Números de Partidas Distintas}
\begin{split}
\#Partidas\ &=\  (\#J-1) * \binom{55}{5} * \binom{50}{5} * \binom{45}{9} * \binom{36}{9} * \binom{27}{9} * \binom{18}{9} * \binom{9}{9}\\
\#Partidas\ &=\ \dfrac{4 * 55!}{2^{27} * 3^{36} * 5^7 * 7^5}\\
\#Partidas\ &=\ \dfrac{2^{52} * 3^{26} * 5^{13} * 7^{8} * 11^{5} * 13^{4} * 17^{3} * 19^{2} * 23^{2} * 29 * 31 * 37 * 41 * 43 * 47 * 53}{2^{27} * 3^{36} * 5^7 * 7^5}\\
\#Partidas\ &=\ \dfrac{2^{25} * 5^{6} * 7^{3} * 11^{5} * 13^{4} * 17^{3} * 19^{2} * 23^{2} * 29 * 31 * 37 * 41 * 43 * 47 * 53}{3^{10}}\\
\#Partidas\ &=\ 560'483'776'167'774'018'942'304'261'616'685'408'000'000\\
\end{split}
\end{equation}

\begin{equation} \label{eq_anos_partidas} \tag{e.q. Tempo de Computação dos Estados}
\begin{split}
Anos\ &=\ \dfrac{N_{partidas\ distintos}}{\nicefrac{partida}{segundo}\times\nicefrac{segundos}{minuto}\times\nicefrac{minutos}{hora}\times\nicefrac{horas}{dia}\times\nicefrac{dias}{ano}}\\
Anos\ &=\ \dfrac{560'483'776'167'774'018'942'304'261'616'685'408'000'000}{\nicefrac{1}{1}\times\nicefrac{60}{1}\times\nicefrac{60}{1}\times\nicefrac{24}{1}\times\nicefrac{365}{1}}\\
Anos\ &=\ 17'772'823'952'554'985'379'956'375'622'041'013\\
\end{split}
\end{equation}

\begin{equation} \label{eq_bytes} \tag{e.q. Bytes na memória}
\begin{split}
Bytes\ &=\ \dfrac{55}{8} + 5 + 5\times 7\\
Bytes\ &=\ 47\ bytes
\end{split}
\end{equation}

\begin{equation} \label{eq_estados} \tag{e.q. Caso Geral}
\begin{split}
\#Estados\ &=\ 2^{55}\times 61\times 5\times (5\times 10 + 2\times 6)\\
\#Estados\ &=\ 2^{55}\times 61\times 5\times 72\\
\#Estados\ &=\ 791'192'382'536'448'737'280
\end{split}
\end{equation}

\begin{equation} \label{eq_anos_estados} \tag{e.q. Tempo de Computação dos Estados}
\begin{split}
Anos\ &=\ \dfrac{N_{estados\ distintos}}{\nicefrac{estado}{segundo}\times\nicefrac{segundos}{minuto}\times\nicefrac{minutos}{hora}\times\nicefrac{horas}{dia}\times\nicefrac{dias}{ano}}\\
Anos\ &=\ \dfrac{791'192'382'536'448'737'280}{\nicefrac{1}{1}\times\nicefrac{60}{1}\times\nicefrac{60}{1}\times\nicefrac{24}{1}\times\nicefrac{365}{1}}\\
Anos\ &=\ 25'088'545'869'369\\
\end{split}
\end{equation}

\begin{equation} \label{eq_poda1} \tag{e.q. Poda 1}
\begin{split}
\#Estados\ &=\ 2^{55}\times 11\times 5\times (5\times 10 + 2\times 6)\\
\#Estados\ &=\ 2^{55}\times 11\times 5\times 72\\
\#Estados\ &=\ 142'674'036'195'097'313'280
\end{split}
\end{equation}
